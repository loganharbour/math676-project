\documentclass[xcolor={usenames,dvipsnames,svgnames,table}, 10pt]{beamer}

\mode<presentation>
\usetheme{Madrid}

\usecolortheme[RGB={80,0,0}]{structure}
\useoutertheme[subsection=false]{miniframes}
\useinnertheme{default}

% hide navigation controlls
\setbeamertemplate{navigation symbols}{}

\setbeamercolor{normal text}{fg=black}
\setbeamercovered{dynamic}
\beamertemplatetransparentcovereddynamicmedium
\setbeamertemplate{caption}[numbered]

\definecolor{Maroon}{RGB}{80,0,0}
\definecolor{BurntOrange}{RGB}{204,85,0}

% load macros and prevent authblk from loading
\input{macros.tex}
\dontusepackage{authblk}

% load packages, settings and definitions
\input{packages.tex}
\input{settings.tex}
\input{definitions.tex}

\usepackage{amsmath}
\usepackage{amssymb}
\usepackage{tikz}
\usepackage{stmaryrd}

% nicer item settings
\setlist[1]{nolistsep,label=\(\textcolor{Maroon}{\blacksquare}\)}
\setlist[2]{nolistsep,label=\(\textcolor{Maroon}{\bullet}\)}

\setenumerate[1]{
	label=\protect\usebeamerfont{enumerate item}%
	\protect\usebeamercolor[fg]{enumerate item}%
	\insertenumlabel.
}

\newcommand{\neutranie}{\tikz[baseline=-0.75ex,black]{
		\draw circle (2mm);
		\node[fill,circle,inner sep=0.5pt] (left eye) at (135:0.8mm) {};
		\node[fill,circle,inner sep=0.5pt] (right eye) at (45:0.8mm) {};
		\draw (-150:0.9mm) -- (-60:1.3mm);
	}
}

\newcommand{\vx}{\mathbf{x}} % x vector
\newcommand{\vo}{\pmb{\Omega}} % omega vector
\newcommand{\vn}{\mathbf{n}} % normal vector
\newcommand{\spatial}{\mathcal{D}} % spatial domain representation
\newcommand{\bd}{\delta \mathcal{D}} % boundary representation
\newcommand{\angular}{\mathcal{S}} % angular domain representation

\title[2D S$_N$ with Diffusion Acceleration]{2D S$_N$ Radiation Transport with Diffusion Acceleration}
\subtitle{MATH 676 -- Final Presentation}
\author[Logan Harbour]{Logan H. Harbour}
\institute[]{Department of Nuclear Engineering \\ Texas A\&M University}
\date[May 1, 2019]

\newcommand{\pinp}[2]{\left(#1,#2\right)}

\begin{document}

{
\setbeamertemplate{headline}[default] 
\begin{frame}
\vfill
\titlepage
\vfill
\begin{figure}[t]
	\centering
	\includegraphics[width=.5\textwidth]{images/nuen}
\end{figure}
\vfill
\end{frame}
}

%%%%%%%%%%%%%%%%%%%%%%%%%%%%%%%%%%%%%%%%%%%%%%%%%%%%%%%%%%%%%%%%%%%%%%%%%%%%%%%%

\begin{frame}\frametitle{Introduction}
	\begin{block}{}
		\begin{itemize}
			\item My Ph.D. work involves various acceleration techniques for method of characteristics (MOC) radiation transport
			\item As an introduction to previous methods, the push for this work is to investigate previously-developed acceleration techniques for S$_N$ radiation transport (as an introduction to my future works)
			\item With this, the first goal was to develop a simple S$_N$ radiation transport code to be accelerated
			\item The second goal was then to utilize a common diffusion-acceleration technique to accelerate said S$_N$ code
			\item Additional goals were added as I was able to complete the first two goals early
		\end{itemize}
	\end{block}
\end{frame}

%%%%%%%%%%%%%%%%%%%%%%%%%%%%%%%%%%%%%%%%%%%%%%%%%%%%%%%%%%%%%%%%%%%%%%%%%%%%%%%%

\begin{frame}\frametitle{One-group Linear Boltzmann Equation}
	Start with the one-group S$_N$ transport equation for a single direction $d$ (neglecting boundary conditions for simplicity), as
	\begin{equation}
		\label{eq:boltzmann}
		\vo_d \cdot \nabla \psi_d(\vx) + \left(\sigma_a(\vx) + \sigma_s(\vx)\right) \psi_d(\vx) - \frac{\sigma_s(\vx)}{2\pi} \sum_{d = 1}^{N_\Omega} \omega_d \psi_d(\vx) = q(\vx)\,,
	\end{equation}
	Let $\mathbb{T}_h$ be the set of all cells of the triangulation in a discontinuous approximation space. The DG weak form with test function $v_d$ is
	\begin{multline}
		\sum_{K \in \mathbb{T}_h} \Big[ \pinp{-\vo_d \cdot \nabla v_d}{\psi_d}_K + \pinp{\psi_d^+ \vo_d \cdot \vn}{v_d}_{\delta K} + \pinp{\sigma_t \psi_d}{v_d}_K  \\ - \pinp{\sigma_s \phi}{v_d}_K = \pinp{q}{v_d}_K\Big]\,,
	\end{multline}
	where $\phi$ is the \textit{scalar flux}, $\phi = \frac{1}{2\pi} \sum_d^{N_\Omega} \omega_d \psi_d$, and $\psi_d^+$ is the upwind value of $\psi_d$ (the value from the side of the face in which $\vo \cdot \vn \geq 0$).
\end{frame}

%%%%%%%%%%%%%%%%%%%%%%%%%%%%%%%%%%%%%%%%%%%%%%%%%%%%%%%%%%%%%%%%%%%%%%%%%%%%%%%%

\begin{frame}\frametitle{Source Iteration}
	To solve, cast Eq. \eqref{eq:boltzmann} with iterative index $\ell$ as
	\begin{equation}
		\label{eq:source-iteation}
		\vo_d \cdot \nabla \psi_d^{(\ell + 1)} + \sigma_t \psi_d^{(\ell + 1)} = \sigma_s \phi^{(\ell)} + q\,,
	\end{equation}
	where $\ell$ is the iterative index, $\psi_d^{(0)} = \phi^{(0)} = \vec{0}$. After solving each direction, $d$, for an iteration $\ell$ in Eq. \eqref{eq:source-iteation}, update the scalar flux with
	\[
		\phi^{(\ell + 1)} = \frac{1}{2\pi} \sum_{d = 1}^{N_\Omega} w_d \psi_d^{(\ell + 1)}\,.
	\]
	As $\sigma_s / \sigma_t \to 1$, particles scatter more before they are absorbed $\rightarrow$ \textbf{\textcolor{Maroon}{the number of source iterations becomes significant!}} 
\end{frame}

%%%%%%%%%%%%%%%%%%%%%%%%%%%%%%%%%%%%%%%%%%%%%%%%%%%%%%%%%%%%%%%%%%%%%%%%%%%%%%%%

\begin{frame}\frametitle{Diffusion Acceleration}
	Simple algebraic manipulations can show that the error in $\psi^{\ell + 1}$ satisfies the transport equation with a source equal to:
	\[
		R^{\ell + 1} = \frac{\sigma_s}{2\pi} (\phi^{\ell + 1} - \phi^\ell)\,.
	\]
	Fourier analysis shows that the angular flux error has a linearly anisotropic angular dependence. The diffusion approximation is exact for such a dependence, therefore we can cast the diffusion problem with the source above to form an error equation with the diffusion approximation that will attenuate the errors most poorly attenuated by the transport solve. The approximation is cast as:
	\begin{equation}
		\label{eq:diffusion}
		-\nabla \cdot D \nabla \delta e^{\ell + 1} + \sigma_a \delta e^{\ell + 1} = \sigma_s \left( \phi^{\ell + 1} - \phi^\ell\right)\,,
	\end{equation}
	where $e^{\ell+1}$ is the approximated error in $\phi^{\ell + 1}$ and $D = 1/3 \sigma_t$.
\end{frame}

%%%%%%%%%%%%%%%%%%%%%%%%%%%%%%%%%%%%%%%%%%%%%%%%%%%%%%%%%%%%%%%%%%%%%%%%%%%%%%%%

\begin{frame}\frametitle{Diffusion Acceleration (cont.)}
	Casting Equation \eqref{eq:diffusion} in the same DG space with interior edges $\mathcal{E}_h^i$ and boundary edges $\mathcal{E}_h^b$ using a modified interior penalty method for the face terms we obtain
	\begin{multline}
		\int_{\mathbb{T}_h} (D \nabla e \cdot \nabla v + \sigma_a e v) + \int_{\mathcal{E}_h^i} \left( \{\!\!\{ D \delta_n e \}\!\!\} \llbracket v \rrbracket + \{\!\!\{ D \delta_n v \}\!\!\} \llbracket e \rrbracket + \kappa \llbracket e \rrbracket \llbracket v \rrbracket \right) \\
		+ \int_{\mathcal{E}_h^b} \left( \kappa e v - D v \delta_n e - D e \delta_n v \right) = \int_{\mathbb{T}_h} (\phi^{\ell + 1} - \phi^\ell) v\,,
	\end{multline}
	where
	\[
		\{\!\!\{ u \}\!\!\} \equiv \frac{u^+ + u^-}{2} \quad \text{and} \quad \llbracket u \rrbracket \equiv u^+ - u^-\,,
	\]
	in which the penalty coefficient is
	\[
		\kappa = \begin{cases} 2 \left(\frac{D^+}{h^+_\bot} + \frac{D^-}{h^-_\bot}\right) & \text{for~interior~edges}\,, \\ 8 \frac{D^-}{h^-_\bot} & \text{for~boundary~edges}\,, \end{cases}
	\]
	and $h^\pm_\bot$ is a characteristic length of the cell in the direction orthogonal to the edge.
\end{frame}

%%%%%%%%%%%%%%%%%%%%%%%%%%%%%%%%%%%%%%%%%%%%%%%%%%%%%%%%%%%%%%%%%%%%%%%%%%%%%%%%

\begin{frame}\frametitle{Diffusion Acceleration Results}
	Consider $\spatial = [0, 10]^2$, $N_\Omega = 20$, $q = 1$, $\sigma_a + \sigma_s = \sigma_t = 100$, and $16^2$ elements. Increase the scattering ratio, $c = \sigma_s / \sigma_t$, with and without diffusion acceleration and observe the results:
	\vfill
	\includegraphics[width=0.95\linewidth]{plots/dsa_residuals}
	\vfill
\end{frame}

%%%%%%%%%%%%%%%%%%%%%%%%%%%%%%%%%%%%%%%%%%%%%%%%%%%%%%%%%%%%%%%%%%%%%%%%%%%%%%%%

\begin{frame}\frametitle{Additional Goals Completed}
	The primarily implementation (transport with diffusion acceleration) was completed earlier than expected. Therefore, additional goals were added (and completed!) to round out the project:
	\vfill
	\begin{block}{Parallel support}
		\begin{itemize}
			\item Supports parallel solves using MPI and Trillinos wrappers, completed primarily by following step-40
			\item Transport is solved with GMRES and the AMG preconditioner
			\item Diffusion is solved with CG and the AMG preconditioner
		\end{itemize}
	\end{block}
	\vfill
	\begin{block}{Reflecting boundary conditions}
		\begin{itemize}
			\item Reflecting boundaries require storing the outgoing flux on the boundaries and then reflecting on the incoming boundaries (bit of a pain, but it works)
			\item Also supported in the diffusion acceleration scheme through adding an additional source term for boundary flux error
		\end{itemize}
	\end{block}
\end{frame}

%%%%%%%%%%%%%%%%%%%%%%%%%%%%%%%%%%%%%%%%%%%%%%%%%%%%%%%%%%%%%%%%%%%%%%%%%%%%%%%%

\begin{frame}\frametitle{Whoop}
	\vfill
	\centering
	\includegraphics[width=0.6\linewidth]{images/atm}
	\vfill
\end{frame}

\end{document}