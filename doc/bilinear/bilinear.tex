\documentclass{article}

\usepackage[margin=1in]{geometry}
\usepackage{graphicx}
\usepackage{amsmath}
\usepackage{amssymb}
\usepackage{color,soul}
\usepackage{mathtools}
\usepackage{physics}

\setlength\parindent{0pt} % No indents
\setlength{\parskip}{1em} % Paragraph skip

\newcommand{\vx}{\mathbf{x}} % x vector
\newcommand{\vo}{\mathbf{\Omega}} % omega vector
\newcommand{\vn}{\mathbf{n}} % normal vector
\newcommand{\spatial}{\mathcal{D}} % spatial domain representation
\newcommand{\boundary}{\delta \mathcal{D}} % boundary representation
\newcommand{\angular}{\mathcal{S}} % angular domain representation

\newcommand{\pageTitle}{MEEN 644 - Homework 3}
\newcommand{\pageAuthor}{Logan Harbour}

\begin{document}

\title{MATH 676 Project -- Bilinear Form}
\author{Logan Harbour}
\date{\today}

\maketitle

Begin with the spatial domain $\spatial \in \mathbb{R}^2$ in which $\boundary$ is on the boundary of $\spatial$. The set of propagation directions $\angular$ is the unit disk.

The linear Boltzmann equation for one-group transport is
\begin{subequations}
	\begin{equation}
		\vo \cdot \nabla \Psi(\vo, \vx) + \sigma_t(\vx) \Psi(\vo, \vx) - \sigma_s(\vx) \Phi(\vx) = q(\vx)\,, \qquad \forall (\vo, \vx) \in \angular \times \spatial\,,
	\end{equation}
	\begin{equation}
		\Phi(\vo, \vx) = \Phi^\text{inc} (\vo, \vx)\,, \qquad \forall(\vo, \vx) \in \angular \times \boundary\,,~ \vo \cdot \vn(x) < 0\,,
	\end{equation}
\end{subequations}
where $\Phi$ is the scalar flux, defined by
\[
	\Phi = \frac{1}{2\pi} \int_{\angular} \Phi(\vo, \vx)~d\Omega\,.
\]

Introduce the $S_N$ discretization by   
\end{document}