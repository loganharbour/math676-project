\documentclass{article}

\usepackage[margin=1in]{geometry}
\usepackage{graphicx}
\usepackage{amsmath}
\usepackage{amssymb}
\usepackage{color,soul}
\usepackage{mathtools}
\usepackage{physics}
\usepackage{titlesec}

\setlength\parindent{0pt} % No indents
\setlength{\parskip}{1em} % Paragraph skip

% Section formatting and spacing
\titleformat{\section}{\normalfont\bfseries}{\thesection}{1em}{}
\titlespacing{\section}{0pt}{0.8ex}{0.2ex}
\titleformat{\subsection}{\normalfont\itshape}{\thesubsection}{1em}{}
\titlespacing{\subsection}{0pt}{0.8ex}{0.2ex}

% Equation numbering by section
\numberwithin{equation}{section}

\newcommand{\vx}{\mathbf{x}} % x vector
\newcommand{\vo}{\mathbf{\Omega}} % omega vector
\newcommand{\vn}{\mathbf{n}} % normal vector
\newcommand{\spatial}{\mathcal{D}} % spatial domain representation
\newcommand{\boundary}{\delta \mathcal{D}} % boundary representation
\newcommand{\angular}{\mathcal{S}} % angular domain representation

\DeclarePairedDelimiterX{\ainp}[2]{\langle}{\rangle}{#1, #2}
\newcommand{\pinp}[2]{\left(#1,#2\right)}

\newcommand{\pageTitle}{MATH 676 Project: S$_N$ Summary}
\newcommand{\pageAuthor}{Logan Harbour}

\begin{document}
	
\title{\MakeUppercase{\normalsize \textbf{\pageTitle}} \vspace{-0.6cm}}
\author{\MakeUppercase{\small \pageAuthor}}
\date{\vspace{-1.2cm}}

\maketitle

\section{Problem definition}

Begin with the spatial domain $\spatial \in \mathbb{R}^2$ in which $\boundary$ is on the boundary of $\spatial$. The set of propagation directions $\angular$ is the unit disk.

The linear Boltzmann equation for one-group transport is
\begin{subequations}
	\label{eq:boltzmann}
	\begin{equation}
		\vo \cdot \nabla \Psi(\vo, \vx) + \sigma_t(\vx) \Psi(\vo, \vx) - \sigma_s(\vx) \Phi(\vx) = q(\vx)\,, \qquad \forall (\vo, \vx) \in \angular \times \spatial\,,
	\end{equation}
	\begin{equation}
		\Phi(\vo, \vx) = \Phi^\text{inc} (\vo, \vx)\,, \qquad \forall(\vo, \vx) \in \angular \times \boundary\,,~ \vo \cdot \vn(\vx) < 0\,,
	\end{equation}
\end{subequations}
where $\Phi$ is the scalar flux, defined by
\[
	\Phi = \frac{1}{2\pi} \int_{\angular} \Phi(\vo, \vx) d\Omega\,.
\]

\section{S$_N$ discretization}

Introduce the S$_N$ discretization, which replaces the angular flux with a discrete angular flux, as
\begin{equation}
	\label{eq:sn_discretization}
	\psi(\vx) = [\psi_1(\vx), \psi_2(\vx), \ldots \psi_{N_\Omega}(\vx)]^T\,.
\end{equation}
We then introduce the quadrature rule $\{ (\vo_d, \omega_d), d = 1, \ldots, N_\Omega\}$ where $\sum_d \omega_d = 2 \pi$. With said quadrature rule, we have 
\[
	\int_\angular f(\vo, \vx) d\Omega \approx \sum_{d = 1}^{N_\Omega} w_d f(\vo_d, \vx)\,.
\]
This discretization allows us to write the system in Equation \eqref{eq:boltzmann} as
\begin{subequations}
	\label{eq:sn_equations}
	\begin{equation}
		\label{eq:sn_equations_domain}
		\vo_d \cdot \nabla \psi_d(\vx) + \sigma_t(\vx) \psi_d(\vx) - \sigma_s(\vx) \phi(\vx) = q(\vx)\,, \qquad \forall \vx \in \spatial
	\end{equation}
	\begin{equation}
		\label{eq:sn_equations_boundary}
		\psi_d(\vx) = \Psi^\text{inc}_j (\vx)\,, \qquad \forall \vx \in \boundary\,,~ \vo_d \cdot \vn(\vx) < 0\,, 
	\end{equation}
\end{subequations}
where the discrete scalar flux, $\phi$, is
\[
	\phi(\vx) = \frac{1}{2\pi} \sum_{d = 1}^{N_\Omega} w_j \psi_j(\vx)\,.
\]

\section{Discontinuous Galerkin discretization}

Define $\mathbb{T}_h$ as the set of all active cells of the triangulation for $\spatial$ and $\mathbb{F}_h$ as the set of all active interior faces. Define a discontinuous approximation space for the scalar flux based on the mesh $\mathbb{T}_h$ as
\begin{equation}
	V_h \in \{ v \in L^2(\spatial)~|~\forall K \in \mathbb{T}_h\,, v|_K \in P_K\}\,,
\end{equation}
where the finite-dimensional space $P_K$ is assumed to contain $\mathbb{P}_k$, the set of polynomials of degree at most $k$. The discrete space for the angular flux consists of copies of $V_h$ for each of the discrete ordinates, as
\begin{equation}
	W_h = \left(V_h\right)^{N_\Omega}\,.
\end{equation}

Multiply Equation \eqref{eq:sn_equations_boundary} by the test function $v_d \in V_h$ and integrate as
\begin{equation}
	\label{eq:fem_integrate}
	\sum_{K \in \mathbb{T}_h} \left[ \pinp{\vo_d \cdot \nabla \psi_d}{v_d}_K + \pinp{\sigma_t \psi_d}{v_d}_K - \pinp{\sigma_s \phi}{v_d}_K = \pinp{q}{v_d}_K\right]\,,
\end{equation}
and integrate the first term by parts to obtain
\begin{equation}
	\label{eq:fem_parts}
	\sum_{K \in \mathbb{T}_h} \left[ \pinp{-\vo_d \cdot \nabla v_d}{\psi_d}_K + \pinp{\psi_d \vo_d \cdot \vn}{v_d}_{\delta K} + \pinp{\sigma_t \psi_d}{v_d}_K - \pinp{\sigma_s \phi}{v_d}_K = \pinp{q}{v_d}_K\right]\,,
\end{equation}
where $\vn$ is the outward normal. Note that the surface integration in Equation \eqref{eq:fem_parts} is double-valued due to the discontinuous approximation. We introduce the upwind approximation
\begin{equation}
	\psi_d \vo_d \cdot \vn = \psi^+_d \vo_d \cdot \vn\,,
\end{equation}
where $\psi_d^+$ is the upwind value of $\psi_d$, that is, the value from the side of the face in which $\vo \cdot \vn \geq 0$. The weak form is then defined as
\begin{equation}
	\label{eq:fem_upwind}
	\sum_{K \in \mathbb{T}_h} \left[ \pinp{-\vo_d \cdot \nabla v_d}{\psi_d}_K + \pinp{\psi_d^+ \vo_d \cdot \vn}{v_d}_{\delta K} + \pinp{\sigma_t \psi_d}{v_d}_K - \pinp{\sigma_s \phi}{v_d}_K = \pinp{q}{v_d}_K\right]\,.
\end{equation}

\end{document}