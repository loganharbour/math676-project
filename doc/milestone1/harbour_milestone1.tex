\documentclass[xcolor={usenames,dvipsnames,svgnames,table}]{beamer}

\mode<presentation>
\usetheme{Madrid}

\usecolortheme[RGB={80,0,0}]{structure}
\useoutertheme[subsection=false]{miniframes}
\useinnertheme{default}

% hide navigation controlls
\setbeamertemplate{navigation symbols}{}

\setbeamercolor{normal text}{fg=black}
\setbeamercovered{dynamic}
\beamertemplatetransparentcovereddynamicmedium
\setbeamertemplate{caption}[numbered]

\definecolor{Maroon}{RGB}{80,0,0}
\definecolor{BurntOrange}{RGB}{204,85,0}

% load macros and prevent authblk from loading
\input{macros.tex}
\dontusepackage{authblk}

% load packages, settings and definitions
\input{packages.tex}
\input{settings.tex}
\input{definitions.tex}

\usepackage{amsmath}
\usepackage{amssymb}
% nicer item settings
\setlist[1]{nolistsep,label=\(\textcolor{Maroon}{\blacksquare}\)}
\setlist[2]{nolistsep,label=\(\textcolor{Maroon}{\bullet}\)}

\setenumerate[1]{
	label=\protect\usebeamerfont{enumerate item}%
	\protect\usebeamercolor[fg]{enumerate item}%
	\insertenumlabel.
}

\newcommand{\vx}{\mathbf{x}} % x vector
\newcommand{\vo}{\pmb{\Omega}} % omega vector
\newcommand{\vn}{\mathbf{n}} % normal vector
\newcommand{\spatial}{\mathcal{D}} % spatial domain representation
\newcommand{\bd}{\delta \mathcal{D}} % boundary representation
\newcommand{\angular}{\mathcal{S}} % angular domain representation

\title[2D S$_N$ with Diffusion Acceleration]{2D S$_N$ with Diffusion Acceleration \\ {\small MATH 676: 1$^\text{st}$ Milestone Presentation}}
\author[L. Harbour]{Logan H. Harbour}
\institute[]{Department of Nuclear Engineering \\ Texas A\&M University}
\date[March 20, 2019]

\begin{document}

{
\setbeamertemplate{headline}[default] 
\begin{frame}
\vfill
\titlepage
\vfill
\begin{figure}[t]
	\centering
	\includegraphics[width=.5\textwidth]{images/nuen}
\end{figure}
\vfill
\end{frame}
}

%%%%%%%%%%%%%%%%%%%%%%%%%%%%%%%%%%%%%%%%%%%%%%%%%%%%%%%%%%%%%%%%%%%%%%
\section{Introduction}
\subsection{}
%%%%%%%%%%%%%%%%%%%%%%%%%%%%%%%%%%%%%%%%%%%%%%%%%%%%%%%%%%%%%%%%%%%%%%

\begin{frame}\frametitle{One-group Linear Boltzmann Equation}
	Define the spatial domain $\spatial \in \mathbb{R}^2$ in which $\bd$ is on the boundary of $\spatial$. The set of propagation directions $\angular$ is the unit disk.
	\begin{subequations}
		\label{eq:boltzmann}
		\begin{multline}
			\vo \cdot \nabla \Psi(\vo, \vx) + \sigma_t(\vx) \Psi(\vo, \vx) -\sigma_s(\vx) \Phi(\vx) = q(\vx)\,, \\ \forall (\vo, \vx) \in \angular \times \spatial\,,
		\end{multline}
		\begin{equation}
		\Phi(\vo, \vx) = \Phi^\text{inc} (\vo, \vx)\,, \qquad \forall(\vo, \vx) \in \angular \times \bd\,,\quad \vo \cdot \vn(\vx) < 0\,,
		\end{equation}
	\end{subequations}
	where $\Phi$ is the scalar flux, defined by
	\[
		\Phi = \frac{1}{2\pi} \int_{\angular} \Phi(\vo, \vx) d\Omega\,.
	\]
\end{frame}

\begin{frame}\frametitle{S$_N$ Discretization}
	Introduce the S$_N$ discretization, which replaces the angular flux with a discrete angular flux, as
	\begin{equation}
		\label{eq:sn_discretization}
		\psi(\vx) = [\psi_1(\vx), \psi_2(\vx), \ldots \psi_{N_\Omega}(\vx)]^T\,.
	\end{equation}
	Introduce a quadrature rule $\{ (\vo_d, \omega_d), d = 1, \ldots, N_\Omega\}$ where $\sum_d \omega_d = 2 \pi$ to cast the linear Boltzmann equation as
	\begin{subequations}
		\label{eq:sn_equations}
		\begin{equation}
		\label{eq:sn_equations_domain}
		\vo_d \cdot \nabla \psi_d(\vx) + \sigma_t(\vx) \psi_d(\vx) - \sigma_s(\vx) \phi(\vx) = q(\vx)\,, \qquad \forall \vx \in \spatial
		\end{equation}
		\begin{equation}
		\label{eq:sn_equations_boundary}
		\psi_d(\vx) = \Psi^\text{inc}_j (\vx)\,, \qquad \forall \vx \in \bd\,,~ \vo_d \cdot \vn(\vx) < 0\,, 
		\end{equation}
	\end{subequations}
	where the discrete scalar flux, $\phi$, is
	\[
		\phi(\vx) = \frac{1}{2\pi} \sum_{d = 1}^{N_\Omega} w_j \psi_j(\vx)\,.
	\]
\end{frame}

\end{document}