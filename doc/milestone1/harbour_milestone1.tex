\documentclass[xcolor={usenames,dvipsnames,svgnames,table}]{beamer}

\mode<presentation>
\usetheme{Madrid}

\usecolortheme[RGB={80,0,0}]{structure}
\useoutertheme[subsection=false]{miniframes}
\useinnertheme{default}

% hide navigation controlls
\setbeamertemplate{navigation symbols}{}

\setbeamercolor{normal text}{fg=black}
\setbeamercovered{dynamic}
\beamertemplatetransparentcovereddynamicmedium
\setbeamertemplate{caption}[numbered]

\definecolor{Maroon}{RGB}{80,0,0}
\definecolor{BurntOrange}{RGB}{204,85,0}

% load macros and prevent authblk from loading
\input{macros.tex}
\dontusepackage{authblk}

% load packages, settings and definitions
\input{packages.tex}
\input{settings.tex}
\input{definitions.tex}

\usepackage{amsmath}
\usepackage{amssymb}
% nicer item settings
\setlist[1]{nolistsep,label=\(\textcolor{Maroon}{\blacksquare}\)}
\setlist[2]{nolistsep,label=\(\textcolor{Maroon}{\bullet}\)}

\setenumerate[1]{
	label=\protect\usebeamerfont{enumerate item}%
	\protect\usebeamercolor[fg]{enumerate item}%
	\insertenumlabel.
}

\newcommand{\vx}{\mathbf{x}} % x vector
\newcommand{\vo}{\pmb{\Omega}} % omega vector
\newcommand{\vn}{\mathbf{n}} % normal vector
\newcommand{\spatial}{\mathcal{D}} % spatial domain representation
\newcommand{\bd}{\delta \mathcal{D}} % boundary representation
\newcommand{\angular}{\mathcal{S}} % angular domain representation

\title[2D S$_N$ with Diffusion Acceleration]{2D S$_N$ with Diffusion Acceleration}
\subtitle{MATH 676 -- Milestone 1 Presentation}
\author[L. Harbour]{Logan H. Harbour}
\institute[]{Department of Nuclear Engineering \\ Texas A\&M University}
\date[March 20, 2019]

\newcommand{\pinp}[2]{\left(#1,#2\right)}

\begin{document}

{
\setbeamertemplate{headline}[default] 
\begin{frame}
\vfill
\titlepage
\vfill
\begin{figure}[t]
	\centering
	\includegraphics[width=.5\textwidth]{images/nuen}
\end{figure}
\vfill
\end{frame}
}

%%%%%%%%%%%%%%%%%%%%%%%%%%%%%%%%%%%%%%%%%%%%%%%%%%%%%%%%%%%%%%%%%%%%%%%%%%%%%%%%

\begin{frame}\frametitle{One-group Linear Boltzmann Equation}
	Begin with the one-group S$_N$ transport equation for a single direction $d$ (neglecting boundary conditions for simplicity), as	
	\begin{equation}
		\label{eq:boltzmann}
		\vo_d \cdot \nabla \psi_d(\vx) + \left(\sigma_a(\vx) + \sigma_s(\vx)\right) \psi_d(\vx) - \frac{\sigma_s(\vx)}{2\pi} \sum_{d = 1}^{N_\Omega} \omega_d \psi_d(\vx) = q(\vx)\,,
	\end{equation}
	where $\sigma_a$ represents a probability of particle absorption and $\sigma_s$ represents a probability of radiation scattering. Let $\mathbb{T}_h$ be the set of all cells of the triangulation in a discontinuous approximation space. The DG weak form with test function $v_d$ is
	\begin{multline}
		\sum_{K \in \mathbb{T}_h} \Big[ \pinp{-\vo_d \cdot \nabla v_d}{\psi_d}_K + \pinp{\psi_d^+ \vo_d \cdot \vn}{v_d}_{\delta K} + \pinp{\sigma_t \psi_d}{v_d}_K  \\ - \pinp{\sigma_s \phi}{v_d}_K = \pinp{q}{v_d}_K\Big]\,,
	\end{multline}
	where $\phi$ is the \textit{scalar flux}, $\phi = \frac{1}{2\pi} \sum_d^{N_\Omega} \omega_d \psi_d$, and $\psi_d^+$ is the upwind value of $\psi_d$ (the value from the side of the face in which $\vo \cdot \vn \geq 0$).
\end{frame}

%%%%%%%%%%%%%%%%%%%%%%%%%%%%%%%%%%%%%%%%%%%%%%%%%%%%%%%%%%%%%%%%%%%%%%%%%%%%%%%%

\begin{frame}\frametitle{Issue: Source Iteration}
	We commonly solve the transport equation by \textit{source iteration}, a form of Richardson iteration. Cast Eq. \eqref{eq:boltzmann} with iterative index $\ell$ as
	\begin{equation}
		\label{eq:source-iteation}
		\vo_d \cdot \nabla \psi_d^{(\ell + 1)} + \sigma_t \psi_d^{(\ell + 1)} = \sigma_s \phi^{(\ell)} + q\,,
	\end{equation}
	where $\ell$ is the iterative index, $\psi_d^{(0)} = \phi^{(0)} = \vec{0}$. After solving each direction, $d$, for an iteration $\ell$ in Eq. \eqref{eq:source-iteation}, update the scalar flux with
	\[
		\phi^{(\ell + 1)} = \frac{1}{2\pi} \sum_{d = 1}^{N_\Omega} w_d \psi_d^{(\ell + 1)}\,.
	\]
	$\psi^{(\ell + 1)}$ is the particles that have scattered at most $\ell$ times. As $\sigma_s / \sigma_t \to 1$, particles scatter more before they are absorbed $\rightarrow$ \textbf{\textcolor{Maroon}{the number of source iterations becomes significant!}} This problem becomes the goal of this work: introduce a diffusion problem as a preconditioner for Eq. \eqref{eq:source-iteation}.
\end{frame}

%%%%%%%%%%%%%%%%%%%%%%%%%%%%%%%%%%%%%%%%%%%%%%%%%%%%%%%%%%%%%%%%%%%%%%%%%%%%%%%%

\begin{frame}\frametitle{Example: Lots of Scattering}
	Introduce $\spatial = [0, 10]^2$, $N_\Omega = 20$, $q = 1$, $\sigma_a + \sigma_s = \sigma_t = 100$, and $64^2$ elements. Increase the scattering ratio, $\sigma_s / \sigma_t$ and observe results.

	\begin{figure}[H]
		\centering
		\begin{subfigure}[t]{0.48\textwidth}
				\includegraphics[width=\linewidth]{plots/scattering_norms}
		\end{subfigure}
		\begin{subfigure}[t]{0.48\textwidth}
			\includegraphics[width=\linewidth]{plots/scattering_times}
		\end{subfigure}
	\end{figure}
\end{frame}

%%%%%%%%%%%%%%%%%%%%%%%%%%%%%%%%%%%%%%%%%%%%%%%%%%%%%%%%%%%%%%%%%%%%%%%%%%%%%%%%

\begin{frame}\frametitle{Diffusion Acceleration}
	\begin{minipage}{0.45\linewidth}
		\begin{itemize}
			\item The source iteration process will converge quickly whenever particles scatter just a few times on average before being absorbed or escaping.
			\item It can converge very slowly in \textit{diffusive} problem, as particles scatter an arbitrary number of times on average before being absorbed or escaping.
		\end{itemize}
	\end{minipage}
	\hfill
	\begin{minipage}{0.50\linewidth}
		\includegraphics[width=\linewidth]{images/eigenvalues}
	\end{minipage}
\end{frame}

%%%%%%%%%%%%%%%%%%%%%%%%%%%%%%%%%%%%%%%%%%%%%%%%%%%%%%%%%%%%%%%%%%%%%%%%%%%%%%%%

\begin{frame}\frametitle{Project Progress and Future Goals}
	\begin{block}{Completed works}
		\begin{itemize}
			\item A one-group, 2D neutron transport code using the S$_N$ approximation has been developed using linear discontinuous finite elements in Deal.ii.
			\item Verified using known constant source solutions and MMS.
			\item Primarily uses the \texttt{MeshWorker} interface as discussed in step-12.
			\item Utilizes \texttt{downstream\_renumbering} to precondition the within-direction solve.
		\end{itemize}
	\end{block}
\end{frame}
\end{document}