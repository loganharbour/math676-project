% Common style and macro defintions for CLASS documents
%=================================================================================================

%%% General Commands

%% structural elements
% An explanation for a function
\newcommand{\explain}[1]{\mbox{\hspace{2em} #1}}
% attemtion box
\newcommand{\attention}[1]{\mbox{\hspace{2em} #1}}
% comments at the sde of the paragraph
\newcommand{\sidenotes}[1]{\marginpar{ {\footnotesize #1} }}
% create empty double page, TODO do we need that?
\newcommand{\clearemptydoublepage}{\newpage{\pagestyle{empty}\cleardoublepage}}
%TODO description here
\newcommand{\apost}{\textit{a posteriori\xspace}}
%TODO description here
\newcommand{\Apost}{\textit{A posteriori}\xspace}

%%% Math
%% Paratnhis etc.
% ()
\newcommand{\parenthesis}[1]{{\left(#1 \right)}}
% []
\newcommand{\bracket}[1]{{\left[ #1 \right]}}
% {}
\newcommand{\bracet}[1]{{\left\{#1 \right\}}}
% <>
\newcommand{\angled}[1]{{\left\langle#1\right\rangle}}

%% common math symbols
% imaginary number
\newcommand{\img}{\ensuremath{\hat{\imath}}\xspace}
% gradient symbol
\newcommand{\grad}{\vec{\nabla}}
\newcommand{\del}{\vec{\nabla}}
% adjoint
\newcommand{\adj}[2][{}]{{{#2}^{\dagger#1}}}
% order number
\newcommand{\order}[1]{^\parenthesis{#1}}
% iteration number
\newcommand{\iter}[1]{^{#1}}

%% common math function
% divergence
\renewcommand{\div}{\del\! \cdot \!}
% rotation
\newcommand{\rot}{\del\! \times \!}
% absolute value
\newcommand{\abs}[1]{\left|#1\right|}
% norm
\newcommand{\norm}[2][{}]{\lVert#2\rVert_{#1}}
% e function
\newcommand{\e}[1]{\mathrm{e}^{#1}}
% power of ten
\newcommand{\tento}[1]{\ensuremath{10^{#1}}\xspace}
% E notation
\newcommand{\E}[1]{\ensuremath{\cdot \tento{#1}}\xspace}
% sign
\DeclareMathOperator{\sign}{sign}

%% fraction
% 1 / 2
\newcommand{\half}[1][1]{\frac{#1}{2}}
% 1 / 3
\newcommand{\third}[1][1]{\frac{#1}{3}}
% 1 / 4
\newcommand{\fourth}[1][1]{\frac{#1}{4}}

% vectors etc
% vector, we use standard latex notation
% matrix
\newcommand{\mat}[1]{\mathbf{#1}}
% tensor
\newcommand{\tensor}[1]{\underline{\underline{#1}}}
% operator symbol
\newcommand{\op}[1]{\mathrm{\mathbf{#1}}}

% derivatives
% first derivatives
% general first derivative, with optional argument for f
\newcommand{\dd}[2][]{\frac{\partial #1}{\partial #2}}
% partial derivative for x, optional argument for f
\newcommand{\ddx}[1][]{\dd[#1]{x}}
% partial derivative for y, optional argument for f
\newcommand{\ddy}[1][]{\dd[#1]{y}}
% partial derivative for z, optional argument for f
\newcommand{\ddz}[1][]{\dd[#1]{z}}
% partial derivative for t, optional argument for f
\newcommand{\ddt}[1][]{\dd[#1]{t}}

% second derivatives, only for a single variable, cross derivatives are too many
% second general derivative, optional argument for f
\newcommand{\ddd}[2][{}]{\frac{\partial^2 #1}{\partial {#2}^2}}
% second partial derivative for x, optional argument for f
\newcommand{\ddxx}[1][]{\ddd[#1]{x}}
% second partial derivative for y, optional argument for f
\newcommand{\ddyy}[1][]{\ddd[#1]{y}}
% second partial derivative for z, optional argument for f
\newcommand{\ddzz}[1][]{\ddd[#1]{z}}
% second partial derivative for t, optional argument for f
\newcommand{\ddtt}[1][]{\ddd[#1]{t}}

% integrals
% integral dx, optional parameter for different variable
\newcommand{\dx}[1][x]{\,d#1}
% integral dy
\newcommand{\dy}{\dx[y]}
% integral dz
\newcommand{\dz}{\dx[z]}
% integral dt
\newcommand{\dt}{\dx[t]}
%integral dmu
\newcommand{\dmu}{\dx[\mu]}
%integral dOmega
\newcommand{\domg}{\dx[\direction]}

% spherical integrals
% shortcut for sphere notation
\newcommand{\sphere}{\ensuremath{\mathcal{S}}\xspace}
% angular quadrature weight
\newcommand{\aqweight}{\omega}

% integral over full sphere
\newcommand{\intsp}{\int_{4\pi}}
% integral over half sphere
\newcommand{\inthalfsp}{\int_{2\pi}}
% polar integral
\newcommand{\intpolar}{\int_{-1}^{1}}
% negative partial polar integral
\newcommand{\intnpolar}{\int_{-1}^{0}}
% positive partial polar integral
\newcommand{\intppolar}{\int_{0}^{1}}


% FEM symbols
% spatial domain
\newcommand*{\domain}{\ensuremath{\mathcal{D}}\xspace}
% boundary of a spatical domain
\newcommand*{\boundary}{\ensuremath{{\partial\domain}}\xspace}
% vacuum boundary of a spatical domain
\newcommand*{\vboundary}{\ensuremath{{\boundary}_v}\xspace}
% reflective boundary of a spatical domain
\newcommand*{\rboundary}{\ensuremath{{\boundary}_r}\xspace}
% interface surface
\newcommand*{\interface}{\ensuremath{{\Gamma}}\xspace}
% general and isotropic test function
\newcommand{\testfct}{\ensuremath{\phi^{*}}\xspace}
% angular test function
\newcommand{\atestfct}{\ensuremath{\psi^{*}}\xspace}
% surface normal
\newcommand{\normal}{\ensuremath{\vec{n}}\xspace}
% boundary normal
\newcommand{\bnormal}{\ensuremath{\normal_\mathrm{b}}\xspace}

% DFEM commands
% interface jump
\newcommand{\jump}[1]{[\![#1]\!]}
% what is the different?
\newcommand{\jmpa}[1]{[\![\![#1]\!]\!]}
% mean value
\newcommand{\meanval}[1]{\{\!\!\{#1\}\!\!\}}

% common physical symbols
% mass stream
\newcommand{\mdot}{\ensuremath{\dot{m}}\xspace}

% nuclear symbols
% Sn
\newcommand{\sn}[1][N]{\ensuremath{S_#1}\xspace}
% Pn
\newcommand{\pn}[1][N]{\ensuremath{P_#1}\xspace}
% keff
\newcommand{\keff}{\ensuremath{k_{\text{eff}}}\xspace}
% kinf
\newcommand{\kinf}{\ensuremath{k_{\text{inf}}}\xspace}

% transport symbols
\newcommand{\addgroup}[1]{\ifthenelse{\isempty{#1}}{}{_{#1}}}
% direction omega
\newcommand{\direction}{\ensuremath{\vec{\Omega}}\xspace}
% direction omega
\newcommand{\position}{\ensuremath{\vec{x}}\xspace}
% current
\newcommand{\current}[1][]{\ensuremath{\vec{J}\addgroup{#1}}\xspace}
% positive half range current
\newcommand{\ppcurrent}[1][]{\ensuremath{\hat{\jmath}\,^+\addgroup{#1}}\xspace}
% negative half range current
\newcommand{\npcurrent}[1][]{\ensuremath{\hat{\jmath}\,^-\addgroup{#1}}\xspace}
% drift vector
\newcommand{\drift}[1][]{\ensuremath{\hat{D}\addgroup{#1}}\xspace}
% local diffusion coefficient
\newcommand{\DC}[1][]{\ensuremath{\mathrm{D}\addgroup{#1}}\xspace}
% nonlocal diffusion tensor
\newcommand{\DCNL}[1][]{\ensuremath{\tensor{\mathrm{D}}\addgroup{#1}}\xspace}
% cross section label
\newcommand{\xslabel}[2][]{\ifthenelse{\isempty{#1}}{\mathrm{#2}}{\mathrm{#2},#1}}
% total cross section
\newcommand{\sigt}[1][]{\ensuremath{\sigma_{\xslabel[#1]{t}}}\xspace}
% scattering cross section
\newcommand{\sigs}[1][]{\ensuremath{\sigma_{\xslabel[#1]{s}}}\xspace}
% fission cross section
\newcommand{\sigf}[1][]{\ensuremath{\sigma_{\xslabel[#1]{f}}}\xspace}
% removal cross section
\newcommand{\sigr}[1][]{\ensuremath{\sigma_{\xslabel[#1]{r}}}\xspace}
% absorption cross section
\newcommand{\siga}[1][]{\ensuremath{\sigma_{\xslabel[#1]{a}}}\xspace}
% transport cross section
\newcommand{\sigtr}[1][]{\ensuremath{\sigma_{\xslabel[#1]{tr}}}\xspace}
% scattering moment
\newcommand{\sigl}[2][{}]{\ensuremath{\sigma_{#2#1}}\xspace}

% total cross section
\newcommand{\Sigt}[1][]{\ensuremath{\Sigma_{\xslabel[#1]{t}}}\xspace}
% scattering cross section
\newcommand{\Sigs}[1][]{\ensuremath{\Sigma_{\xslabel[#1]{s}}}\xspace}
% fission cross section
\newcommand{\Sigf}[1][]{\ensuremath{\Sigma_{\xslabel[#1]{f}}}\xspace}
% removal cross section
\newcommand{\Sigr}[1][]{\ensuremath{\Sigma_{\xslabel[#1]{r}}}\xspace}
% absorption cross section
\newcommand{\Siga}[1][]{\ensuremath{\Sigma_{\xslabel[#1]{a}}}\xspace}
% transport cross section
\newcommand{\Sigtr}[1][]{\ensuremath{\Sigma_{\xslabel[#1]{tr}}}\xspace}
% scattering moment
\newcommand{\Sigl}[2][{}]{\ensuremath{\Sigma_{#2#1}}\xspace}

% spatial weight function
\newcommand{\weight}[1][]{\ensuremath{w\addgroup{#1}}\xspace}

% physical units
% general style for a unit symbol
\newcommand{\unit}[1]{\mathrm{#1}}
% distance
% meter maybe \meter better and more clear?
\newcommand{\m}{\,\unit{m}\xspace}
% centimeter
\newcommand{\cm}{\,\unit{cm}\xspace}
% milimeter
\newcommand{\mm}{\,\unit{mm}\xspace}

% time
% second
\newcommand{\s}{\,\unit{s}\xspace}

% transport units
% scalar flux
\newcommand{\sfluxunit}{\,\ensuremath{\frac{1}{\unit{cm}^2\unit{s}}}\xspace}
% angular flux
\newcommand{\afluxunit}{\,\ensuremath{\frac{1}{\unit{cm}^2\unit{s}\cdot\unit{st}}}\xspace}
% diffusion coefficient
\newcommand{\dcunit}{\,\ensuremath{\unit{cm}}\xspace}


% \newenvironment{myverbatim}%            To change the pseudocode font
% {\par\noindent%
%  \rule[0pt]{\linewidth}{0.2pt}
%  \vspace*{-9pt}
%  \linespread{0.0}\small\verbatim}%
% {\rule[-5pt]{\linewidth}{0.2pt}\endverbatim}
%
% \newenvironment{myverbatim1}%            To change the pseudocode font
% {\par\noindent%
%  \rule[0pt]{\linewidth}{0.2pt}
%  \vspace*{-9pt}
%  \linespread{1.0}\scriptsize\verbatim}%
% {\rule[-5pt]{\linewidth}{0.2pt}\endverbatim}
