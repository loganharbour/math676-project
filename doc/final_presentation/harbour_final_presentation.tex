\documentclass[xcolor={usenames,dvipsnames,svgnames,table}, 10pt]{beamer}

\mode<presentation>
\usetheme{Madrid}

\usecolortheme[RGB={80,0,0}]{structure}
\useoutertheme[subsection=false]{miniframes}
\useinnertheme{default}

% hide navigation controlls
\setbeamertemplate{navigation symbols}{}

\setbeamercolor{normal text}{fg=black}
\setbeamercovered{dynamic}
\beamertemplatetransparentcovereddynamicmedium
\setbeamertemplate{caption}[numbered]

\definecolor{Maroon}{RGB}{80,0,0}
\definecolor{BurntOrange}{RGB}{204,85,0}

% load macros and prevent authblk from loading
\input{macros.tex}
\dontusepackage{authblk}

% load packages, settings and definitions
\input{packages.tex}
\input{settings.tex}
\input{definitions.tex}

\usepackage{amsmath}
\usepackage{amssymb}
\usepackage{tikz}
\usepackage{stmaryrd}
\usepackage{wasysym}

% nicer item settings
\setlist[1]{nolistsep,label=\(\textcolor{Maroon}{\blacksquare}\)}
\setlist[2]{nolistsep,label=\(\textcolor{Maroon}{\bullet}\)}

\setenumerate[1]{
	label=\protect\usebeamerfont{enumerate item}%
	\protect\usebeamercolor[fg]{enumerate item}%
	\insertenumlabel.
}

\newcommand{\neutranie}{\tikz[baseline=-0.75ex,black]{
		\draw circle (2mm);
		\node[fill,circle,inner sep=0.5pt] (left eye) at (135:0.8mm) {};
		\node[fill,circle,inner sep=0.5pt] (right eye) at (45:0.8mm) {};
		\draw (-150:0.9mm) -- (-60:1.3mm);
	}
}

\newcommand{\vx}{\mathbf{x}} % x vector
\newcommand{\vo}{\pmb{\Omega}} % omega vector
\newcommand{\vn}{\mathbf{n}} % normal vector
\newcommand{\spatial}{\mathcal{D}} % spatial domain representation
\newcommand{\bd}{\delta \mathcal{D}} % boundary representation
\newcommand{\angular}{\mathcal{S}} % angular domain representation

\title[S$_N$ with Diffusion Acceleration]{Discrete Ordinates Radiation Transport with\\Diffusion Acceleration\\[0.3cm]}
\subtitle{MATH 676 -- Final Presentation}
\author[Logan Harbour]{Logan H. Harbour}
\institute[]{Department of Nuclear Engineering \\ Texas A\&M University}
\date[May 1, 2019]

\newcommand{\pinp}[2]{\left(#1,#2\right)}

\begin{document}

{
\setbeamertemplate{headline}[default] 
\begin{frame}
\vfill
\titlepage
\vfill
\begin{figure}[t]
	\centering
	\includegraphics[width=.5\textwidth]{images/nuen}
\end{figure}
\vfill
\end{frame}
}

%%%%%%%%%%%%%%%%%%%%%%%%%%%%%%%%%%%%%%%%%%%%%%%%%%%%%%%%%%%%%%%%%%%%%%%%%%%%%%%%

\begin{frame}\frametitle{Introduction}
	\vfill
	\begin{block}{Background}
		\begin{itemize} 
			\setlength\itemsep{0.2em}
			\item My Ph.D. work involves various acceleration techniques for method of characteristics (MOC) radiation transport
			\item Some acceleration methods for discrete ordinates (S$_N$) transport are similar to those in MOC transport
			\item In specific, the focus is diffusion synthetic acceleration (DSA), which attenuates the errors most poorly attenuated by source iteration
		\end{itemize}
	\end{block}
	\vfill
	\begin{block}{Goals}
		\begin{itemize}
			\setlength\itemsep{0.2em}
			\item Develop a one-group, DGFEM, S$_N$ radiation transport code
			\item Develop a one-group, DGFEM, diffusion radiation transport code
			\item Accelerate the S$_N$ source iterations with DSA using the diffusion code
			\item More... if time permits (which it did!)
		\end{itemize}
	\end{block}
	\vfill
\end{frame}

%%%%%%%%%%%%%%%%%%%%%%%%%%%%%%%%%%%%%%%%%%%%%%%%%%%%%%%%%%%%%%%%%%%%%%%%%%%%%%%%

\begin{frame}\frametitle{One-group Linear Boltzmann Equation}
	Start with the one-group S$_N$ transport equation for a single direction $d$ (neglecting boundary conditions for simplicity), as
	\begin{equation}
		\label{eq:boltzmann}
		\vo_d \cdot \nabla \psi_d(\vx) + \left(\sigma_a(\vx) + \sigma_s(\vx)\right) \psi_d(\vx) - \frac{\sigma_s(\vx)}{2\pi} \sum_{d = 1}^{N_\Omega} \omega_d \psi_d(\vx) = q(\vx)\,,
	\end{equation}
	Let $\mathbb{T}_h$ be the set of all cells of the triangulation in a discontinuous approximation space. The DG weak form with test function $v_d$ is
	\begin{multline}
		\sum_{K \in \mathbb{T}_h} \Big[ \pinp{-\vo_d \cdot \nabla v_d}{\psi_d}_K + \pinp{\psi_d^+ \vo_d \cdot \vn}{v_d}_{\delta K} + \pinp{\sigma_t \psi_d}{v_d}_K  \\ - \pinp{\sigma_s \phi}{v_d}_K = \pinp{q}{v_d}_K\Big]\,,
	\end{multline}
	where $\phi$ is the \textit{scalar flux}, $\phi = \sum_d^{N_\Omega} w_d \psi_d$, and $\psi_d^+$ is the upwind value of $\psi_d$ (the value from the side of the face in which $\vo \cdot \vn \geq 0$).
\end{frame}

%%%%%%%%%%%%%%%%%%%%%%%%%%%%%%%%%%%%%%%%%%%%%%%%%%%%%%%%%%%%%%%%%%%%%%%%%%%%%%%%

\begin{frame}\frametitle{Source Iteration}
	To solve, cast Eq. \eqref{eq:boltzmann} with iterative index $\ell$ as
	\begin{equation}
		\label{eq:source-iteation}
		\vo_d \cdot \nabla \psi_d^{(\ell + 1)} + \sigma_t \psi_d^{(\ell + 1)} = \sigma_s \phi^{(\ell)} + q\,,
	\end{equation}
	where $\ell$ is the iterative index, $\psi_d^{(0)} = \phi^{(0)} = \vec{0}$. After solving each direction, $d$, for an iteration $\ell$ in Eq. \eqref{eq:source-iteation}, update the scalar flux with
	\[
		\phi^{(\ell + 1)} = \sum_{d = 1}^{N_\Omega} w_d \psi_d^{(\ell + 1)}\,.
	\]
	As $\sigma_s / \sigma_t \to 1$, particles scatter more before they are absorbed $\rightarrow$ \textbf{\textcolor{Maroon}{the number of source iterations becomes significant!}} 
\end{frame}

%%%%%%%%%%%%%%%%%%%%%%%%%%%%%%%%%%%%%%%%%%%%%%%%%%%%%%%%%%%%%%%%%%%%%%%%%%%%%%%%

\begin{frame}\frametitle{Diffusion Acceleration}
	Simple algebraic manipulations can show that the error in $\psi^{\ell + 1}$ satisfies the transport equation with a source equal to:
	\[
		R^{\ell + 1} = \frac{\sigma_s}{2\pi} (\phi^{\ell + 1} - \phi^\ell)\,.
	\]
	Fourier analysis shows that the angular flux error has a linearly anisotropic angular dependence. The diffusion approximation is exact for such a dependence $\rightarrow$ cast the diffusion problem to form an error equation with the diffusion approximation that will attenuate the errors most poorly attenuated by the transport solve.\\~
	
	Said approximation is cast as:
	\begin{equation}
		\label{eq:diffusion}
		-\nabla \cdot D \nabla e^{\ell + 1} + \sigma_a e^{\ell + 1} = \sigma_s \left( \phi^{\ell + 1} - \phi^\ell\right)\,,
	\end{equation}
	where $e^{\ell+1}$ is the approximated error in $\phi^{\ell + 1}$ and $D = 1/3 \sigma_t$.
\end{frame}

%%%%%%%%%%%%%%%%%%%%%%%%%%%%%%%%%%%%%%%%%%%%%%%%%%%%%%%%%%%%%%%%%%%%%%%%%%%%%%%%

\begin{frame}\frametitle{Diffusion Acceleration (cont.)}
	Casting Equation \eqref{eq:diffusion} in the same DG space with interior edges $\mathcal{E}_h^i$ and boundary edges $\mathcal{E}_h^b$ using a modified interior penalty method for the face terms we obtain
	\begin{multline}
		\int_{\mathbb{T}_h} (D \nabla e \cdot \nabla v + \sigma_a e v) + \int_{\mathcal{E}_h^i} \left( \{\!\!\{ D \delta_n e \}\!\!\} \llbracket v \rrbracket + \{\!\!\{ D \delta_n v \}\!\!\} \llbracket e \rrbracket + \kappa \llbracket e \rrbracket \llbracket v \rrbracket \right) \\
		+ \int_{\mathcal{E}_h^b} \left( \kappa e v - D v \delta_n e - D e \delta_n v \right) = \int_{\mathbb{T}_h} (\phi^{\ell + 1} - \phi^\ell) v\,,
	\end{multline}
	where
	\[
		\{\!\!\{ u \}\!\!\} \equiv \frac{u^+ + u^-}{2} \quad \text{and} \quad \llbracket u \rrbracket \equiv u^+ - u^-\,,
	\]
	in which the penalty coefficient is
	\[
		\kappa = \begin{cases} 2 \left(\frac{D^+}{h^+_\bot} + \frac{D^-}{h^-_\bot}\right) & \text{for~interior~edges}\,, \\ 8 \frac{D^-}{h^-_\bot} & \text{for~boundary~edges}\,, \end{cases}
	\]
	and $h^\pm_\bot$ is a characteristic length of the cell in the direction orthogonal to the edge.
\end{frame}

%%%%%%%%%%%%%%%%%%%%%%%%%%%%%%%%%%%%%%%%%%%%%%%%%%%%%%%%%%%%%%%%%%%%%%%%%%%%%%%%

\begin{frame}\frametitle{Constant Solution Verification}
	\vfill
	Consider $\spatial = [0, 10]^2$, $N_\Omega = 4$, $q = 1$, $\sigma_t = 10$, and $64^2$ elements. Top and right boundary conditions are reflective. Bottom and left boundary conditions are either vacuum or incident isotropic flux of $q / \sigma_a$.
	\vfill
	\centering
	\includegraphics[width=0.9\linewidth]{plots/constant_solution}
	\vfill
\end{frame}

%%%%%%%%%%%%%%%%%%%%%%%%%%%%%%%%%%%%%%%%%%%%%%%%%%%%%%%%%%%%%%%%%%%%%%%%%%%%%%%%

\begin{frame}\frametitle{Diffusion Acceleration Results}
	\vfill
	Consider $\spatial = [0, 10]^2$, $N_\Omega = 20$, $q = 1$, $\sigma_a + \sigma_s = \sigma_t = 100$, and $16^2$ elements. Increase the scattering ratio, $c = \sigma_s / \sigma_t$, with and without diffusion acceleration and observe the results:
	\vfill
	\centering
	\includegraphics[width=0.9\linewidth]{plots/dsa_residuals}
	\vfill
\end{frame}

%%%%%%%%%%%%%%%%%%%%%%%%%%%%%%%%%%%%%%%%%%%%%%%%%%%%%%%%%%%%%%%%%%%%%%%%%%%%%%%%

\begin{frame}\frametitle{Additional Goals Completed}
	The primarily implementation (transport with diffusion acceleration) was completed earlier than expected. Therefore, additional goals were added to round out the project:
	\vfill
	\begin{block}{Parallel support}
		\begin{itemize}
			\setlength\itemsep{0.2em}
			\item Supports parallel solves using MPI and Trillinos wrappers, completed primarily by following step-40
			\item Transport is solved with GMRES and the AMG preconditioner
			\item Diffusion is solved with CG and the AMG preconditioner
		\end{itemize}
	\end{block}
	\vfill
	\begin{block}{Reflecting boundary conditions}
		\begin{itemize}
			\setlength\itemsep{0.2em}
			\item Reflecting boundaries require storing the outgoing flux on the boundaries and then reflecting on the incoming boundaries (bit of a pain, but it works)
			\item Also supported in the diffusion acceleration scheme through adding an additional source term for boundary flux error
		\end{itemize}
	\end{block}
\end{frame}

%%%%%%%%%%%%%%%%%%%%%%%%%%%%%%%%%%%%%%%%%%%%%%%%%%%%%%%%%%%%%%%%%%%%%%%%%%%%%%%%

\begin{frame}\frametitle{Conclusions}
	\vfill
	\centering
	\begin{minipage}{0.4\linewidth}
		\centering
		\includegraphics[width=0.9\linewidth]{images/atm}
	\end{minipage}
	\begin{minipage}{0.59\linewidth}
		\begin{itemize}
			\setlength\itemsep{0.2em}
			\item S$_N$ and diffusion codes completed as desired
			\item Diffusion acceleration was completed
			\item Both were verified using simple test cases
			\item Additional goals were met: parallelism, reflecting boundary conditions
			\item Had a lot of fun digging through Deal.ii tutorials and Doxygen
			\item Looking forward to using this code as a simple test bed in the future
		\end{itemize}
	\end{minipage}
	\vfill
	\centering
	{\Large Thank you for your time \smiley{}}
\end{frame}

\end{document}